% https://nymity.ch/tikz/

\documentclass[hyperref={pdfpagelabels=true},table,dvipsnames,14pt,aspectratio=169]{beamer}
\usetheme{Boadilla}
\setbeamertemplate{navigation symbols}{}

\usepackage{amsthm, amsfonts, amssymb, amsmath, graphicx}
\usepackage{amsmath}
\usepackage{mathtools}
\usepackage{amsfonts, txfonts}
\usepackage{wasysym}
\usepackage{xcolor}
\usepackage{caption}
\usepackage{comment}
\usepackage{cryptocode}
\urlstyle{same}
\usepackage{framed}
\FrameSep5pt

\usepackage{multicol}

%\usepackage[flushmargin]{footmisc}
%\setlength\footnotemargin{1em}

\usepackage{graphicx}
\graphicspath{ {images/} }

\usepackage{tikz}
\usepackage{tikzpeople}
\usepackage{pgfplots}
\usetikzlibrary{arrows}

\usetikzlibrary{positioning}
\usetikzlibrary{calc}
\usetikzlibrary{shapes}
\usetikzlibrary{shapes.callouts}

\usepackage{capt-of}
\beamertemplatenavigationsymbolsempty
\setbeamercovered{transparent}


\newcommand{\dan}[1]{ { \color{cyan}  Dan:  #1 } }
\newcommand{\chelsea}[1]{ { \color{violet}  Chelsea: #1 } }


\newcommand{\cognac}{Cognac\xspace}
\newcommand{\cognaclim}{Cognac\hbox{-}Limited\xspace}

\newcommand{\defn}[1]{{\bf {#1}}}

\newcommand\supth{\ensuremath{^{\textrm{\scriptsize th}}}}
\newcommand{\abs}[1]{\left\lvert#1\right\rvert}
\newcommand{\bigabs}[1]{\bigl| #1 \bigr|}
\newcommand{\iseq}{\stackrel{?}{=}}
\newcommand{\zo}{\{ 0, 1\}}

\newcommand{\id}{\mathsf{id}}

\newcommand{\ass}{\mathsf{assumption}}
\newcommand{\digest}{\mathsf{digest}}


% prelim
\newcommand{\MM}{\mathcal{M}}

% games
\newcommand{\BigO}{\mathcal{O}}
\newcommand{\oracle}{\BigO}
\newcommand{\G}{\mathbb{G}}
\newcommand{\GG}{\G}
\newcommand{\gpar}{\mathcal{G}}
\newcommand{\F}{\mathbb{F}}
\newcommand{\main}{\mathsc{main}}
\newcommand{\secp}{\lambda}
\newcommand{\usecp}{1^{\secp}}
\newcommand{\rgets}{\hskip2.3pt{\gets\!\!\mbox{\scriptsize${\$}$\normalsize}}\,}
\newcommand{\randpick}{\rgets}
\newcommand{\biased}{\mathsf{biased}}
\newcommand{\notbiased}{\mathsf{uniform}}
\newcommand{\random}{\rgets}
\newcommand{\pr}{\mathsf{Pr}}
\newcommand{\adv}{\mathcal{A}}
\newcommand{\negl}{\nu}
\newcommand{\advant}[2]{\ensuremath{\textsf{Adv}^{\textrm{#1}}_{#2}}}


\newcommand{\transcript}{\mathsf{trans}}
\newcommand{\nonce}{\mathsf{nonce}}
\newcommand{\session}{\mathsf{\mu}}
\newcommand{\xcoord}{\tau}

\newcommand{\Ind}{W}
\newcommand{\B}{\mathcal{B}}
\newcommand{\C}{\mathcal{C}}
\newcommand{\X}{\mathcal{X}}
\newcommand{\Sign}{\mathsf{Sign}}
\newcommand{\Hash}{\mathsf{H}}
\newcommand{\Hrho}{\mathsf{H}_{\rho}}
\newcommand{\Hcom}{\mathsf{H}_{c}}
%\newcommand{\Hpop}{\mathsf{H}_{pop}}
\newcommand{\Hblind}{\mathsf{H}_{bl}}
\newcommand{\Qrho}{\mathsf{Q}_{\rho}}
\newcommand{\Qpop}{\mathsf{Q}_{pop}}
\newcommand{\Qblind}{\mathsf{Q}_{bl}}

\newcommand{\state}{st}
\newcommand{\honest}{\mathsf{honest}}
\newcommand{\corrupt}{\mathsf{corrupt}}

% secret sharing
\newcommand{\shamir}{\mathsf{Shamir}}
\newcommand{\issueshares}{\mathsf{Share}}
\newcommand{\deterministicshare}{\mathsf{DeterministicShare}}
\newcommand{\recover}{\mathsf{Recover}}
\newcommand{\recoveryset}{\mathsf{R}}
\newcommand{\commit}{\mathsf{Commit}}
\newcommand{\verify}{\mathsf{Verify}}
\newcommand{\prove}{\mathsf{Prove}}
\newcommand{\vsscommit}{\vec{C}}
\newcommand{\vsscommitone}{\vec{D}}
\newcommand{\vsscommittwo}{\vec{E}}
\newcommand{\identifycheaters}{\mathsf{IdentifyCheaters}}

% PVPRF
\newcommand{\cvprf}{\mathsf{CVPRF}}
\newcommand{\extract}{\mathsf{Extract}}
\newcommand{\derive}{\mathsf{Derive}}
\newcommand{\witness}{\mathit{w}}
\newcommand{\commitment}{\mathit{C}}
\newcommand{\commitmenthash}{\bar{c}}
\newcommand{\blinding}{\psi}
\newcommand{\witnessspace}{\mathcal{W}}
\newcommand{\blindingspace}{\mathcal{B}}

% DKG
\newcommand{\dkg}{\mathcal{D}}
\newcommand{\keygen}{\mathsf{KeyGen}}
\newcommand{\qual}{\mathsf{qual}}

\newcommand{\polyone}{\textit{f}}
\newcommand{\polytwo}{\textit{h}}
\newcommand{\polythree}{\textit{h}'}


% algorithms
\newcommand{\Setup}{\mathsf{Setup}}
\newcommand{\ParGen}{\mathsf{Setup}}
\newcommand{\pp}{\mathsf{par}}
\newcommand{\KeyGen}{\mathsf{KeyGen}}

\newcommand{\GJKRAttack}{Key\hbox{-}Influence\xspace} % naming for GJKR attack
\newcommand{\GJKRConstruction}{GJKR\xspace} % naming for GJKR construction

\newcommand{\Z}{\mathbb{Z}}
\newcommand{\Zq}{\Z_q}
\newcommand{\Zp}{\Zq}
\newcommand{\pk}{\mathit{pk}}
\newcommand{\PK}{\pk}
\newcommand{\sk}{\mathit{sk}}
\newcommand{\SK}{\sk}
\newcommand{\secretshare}{\mathit{s}}
\newcommand{\secretshareone}{\mathit{u}}
\newcommand{\secretsharetwo}{\mathit{v}}
\newcommand{\secret}{\gammma}
\newcommand{\secretone}{\alpha}
\newcommand{\secrettwo}{\beta}
\newcommand{\blindingsum}{\phi}

% Threshold notation
\newcommand{\shareone}{\bar{u}}
\newcommand{\sharetwo}{\bar{v}}

\newcommand{\lagrange}{\lambda}
\newcommand{\DKG}{\mathsf{DKG}}
\newcommand{\SIM}{\mathsf{SIM}}
\newcommand{\simid}{\ell}

% construction two
\newcommand{\bcomm}{\delta}
\newcommand{\dkgchal}{Y}
\newcommand{\win}{y}
\newcommand{\indeter}{Z}


% games
\newcommand{\algcom}{\em\ \ /\!/ \  }

\newcommand{\Perfect}{Perfect}
\newcommand{\Bounded}{Bounded}
\newcommand{\Relative}{Relative}




% Method to add an item that will appear in every slide
%\addtobeamertemplate{footline}{%
%
%  \tikz[remember picture, overlay] \fill[blue] (current page.south east)
%    rectangle ++(1cm,2cm);
%
%}{}


\pgfdeclarelayer{background}
\pgfdeclarelayer{backbackground}
\pgfdeclarelayer{foreground}
\pgfsetlayers{backbackground,background,main,foreground}   %% some additional layers for demo

\usetikzlibrary{shapes,decorations.shapes}
\tikzset{>=latex}



\makeatletter
\pgfdeclareshape{document}{
\inheritsavedanchors[from=rectangle] % this is nearly a rectangle
\inheritanchorborder[from=rectangle]
\inheritanchor[from=rectangle]{center}
\inheritanchor[from=rectangle]{north}
\inheritanchor[from=rectangle]{south}
\inheritanchor[from=rectangle]{west}
\inheritanchor[from=rectangle]{east}
% ... and possibly more
\backgroundpath{% this is new
% store lower right in xa/ya and upper right in xb/yb
\southwest \pgf@xa=\pgf@x \pgf@ya=\pgf@y
\northeast \pgf@xb=\pgf@x \pgf@yb=\pgf@y
% compute corner of ‘‘flipped page’’
\pgf@xc=\pgf@xb \advance\pgf@xc by-10pt % this should be a parameter
\pgf@yc=\pgf@yb \advance\pgf@yc by-10pt
% construct main path
\pgfpathmoveto{\pgfpoint{\pgf@xa}{\pgf@ya}}
\pgfpathlineto{\pgfpoint{\pgf@xa}{\pgf@yb}}
\pgfpathlineto{\pgfpoint{\pgf@xc}{\pgf@yb}}
\pgfpathlineto{\pgfpoint{\pgf@xb}{\pgf@yc}}
\pgfpathlineto{\pgfpoint{\pgf@xb}{\pgf@ya}}
\pgfpathclose
% add little corner
\pgfpathmoveto{\pgfpoint{\pgf@xc}{\pgf@yb}}
\pgfpathlineto{\pgfpoint{\pgf@xc}{\pgf@yc}}
\pgfpathlineto{\pgfpoint{\pgf@xb}{\pgf@yc}}
\pgfpathlineto{\pgfpoint{\pgf@xc}{\pgf@yc}}
}
}
\makeatother


\usepackage{filecontents}

\begin{filecontents}{msr-start.bib}



\end{filecontents}




\title{Attacks and Fixes on \\ Distributed Key Generation Protocols}
\author{Chelsea Komlo, University of Waterloo}
\date[November 2021]{ Cryptography Reading Group, November 2021}

\begin{document}
\setbeamertemplate{itemize items}[triangle]

\tikzstyle{doc}=[%
draw,
thick,
align=center,
color=black,
shape=document,
minimum width=12mm,
minimum height=15.2mm,
shape=document,
inner sep=2ex,
]

\begin{frame}
        \thispagestyle{empty}
        \maketitle
\end{frame}

\begin{frame}
  \huge
  \centering
  Background
\end{frame}

\begin{frame}
  \frametitle{Shamir Secret Sharing}

  \begin{itemize}
    \item<1-> Allows a \emph{dealer} to share a secret $\secretone$ among $n$ participants, where $t$ participants must cooperate to recover $\secretone$.
    \item[]
    \item<2-> $f$ is the polynomial defined by the coefficients

      \[f = \secretone + a_1 x + a_2 x^2 + a_3 x^3 + \ldots + a_{t-1}x^{t-1}  \]

    \item[]
    \item<3-> Each participant $i \in \{1, \ldots, n \}$ receives a share $\shareone_i \gets f(i) $.
    \item[]
    \item<4-> Recovering $\secretone$ is done by polynomial interpolation.
  \end{itemize}
\end{frame}

\begin{frame}
  \frametitle{Verifiable Secret Sharing}

  \begin{itemize}
    \item<1-> Allows participants to verify their share $\shareone_i = f(i)$ is on the same polynomial as all other participants.
    \item[]
    \item<2-> $\vsscommitone \gets \langle A_0, A_1, \ldots, A_{t-1}  \rangle$, where

      \[ A_0 \gets g^{\secretone},\ A_1 \gets g^{a_1}, \ldots \]
    \item[]
    \item<3-> Perform polynomial interpolation in the exponent to check that $g^{f(i)}$ is a point on $g^f$.
  \end{itemize}
\end{frame}

\begin{frame}
  \frametitle{Distributed Key Generation (DKGs)}

  \begin{itemize}
    \item<1-> $n$ parties participate, all equally trusted.
    \item[]
    \item<2-> $\mathit{KeyGen}(\lambda, n, t) \rightarrow (\pk, \{\sk_1, \ldots, \sk_n \})$: $n$ parties cooperate to generate a public key $\pk$ and $n$ secret key shares.
    \item[]
    \item<3-> $\mathit{Recover}(\{ \sk_i \}_C) \rightarrow \sk$: Assuming $\lvert C \rvert \geq t$,
      $\sk$ can be recovered by combining $\{ \sk_i \}_C$.
  \end{itemize}
\end{frame}

\begin{frame}
  \frametitle{Use Cases for DKGs}

  \begin{itemize}
    \item<1-> Key generation for threshold signatures
    \item[]
    \item<2-> Distributed PRFs
    \item[]
    \item<3-> Anonymous token issuance (Privacy Pass)
  \end{itemize}
\end{frame}

\begin{frame}
  \frametitle{Correctness of a DKG}

  \begin{itemize}
    \item<1-> \textbf{(C1)}: All subsets of $t$ shares define the same secret key $\sk$ (or any subset fulfilling the required access structure).
\let\thefootnote\relax\footnote{
Gennaro, Rosario and Jarecki, Stanislaw and Krawczyk, Hugo and Rabin, Tal. Secure Distributed Key Generation for Discrete-Log Based Cryptosystems. Journal of Cryptology, 2007
}
    \item[]
    \item<2-> \textbf{(C2)}: All parties that honestly followed the protocol have the same value of the public key $\pk$.
  \end{itemize}
\end{frame}

\begin{frame}
  \frametitle{Security: Two Approaches}

  \begin{itemize}
    \item<1-> \emph{Secrecy}: Nothing about $\sk$ is revealed beyond what is revealed by $\pk$.
      Or, there exists a simulator that can simulate the protocol to an adversary for a challenge public key $\pk$.
    \item[]
    \item<2-> \emph{Relative Security}: Prove the security of the DKG in the context of demonstrating security of another protocol.
  \footnote{
  Kobi Gurkan, Philipp Jovanovic, Mary Maller, Sarah Meiklejohn, Gilad Stern, Alin Tomescu.
  Aggregatable Distributed Key Generation, EUROCRYPT 2021 }
  \end{itemize}
\end{frame}

\begin{frame}
  \frametitle{Requirement of Security: Uniform Distribution}

  \begin{itemize}
    \item<1-> In a single-party setting, ensuring that a secret key is randomly sampled is easy.
    \item[]
    \item<2-> In a multi-party setting, where the adversary participates in key generation,
      ensuring the output key is uniformly distributed is harder.
  \end{itemize}
\end{frame}


\begin{frame}
  \frametitle{Prior Work (Non-Pairing Based DKGs)}

  \begin{itemize}
    \item<1-> Pedersen DKG
  \begin{itemize}
      \item<2-> GJKR showed that an adversary can bias key material, but by a limited amount.
      \item<3-> Relative proof of security, in the context for Schnorr threshold signing.
  \footnote{
  Rosario Gennaro, Stanislaw Jarecki, Hugo Krawczyk, Tal Rabin.
  Secure Applications of Pedersen's Distributed Key Generation Protocol,
  CT-RSA, 2003.}
  \end{itemize}
    \item[]
    \item<4-> GJKR
  \begin{itemize}
      \item<5-> Adds an additional round to ensure protection against key bias.
      \item<6-> Proof of secrecy.
  \end{itemize}
  \end{itemize}
\end{frame}



\begin{frame}
  \huge
  \centering
  Key-Bias Attack against Pedersen DKG
\end{frame}

\begin{frame}
  \centering
\begin{tikzpicture}
\node at (0,0) {
\scalebox{0.9} {
  \procedure{Pedersen DKG}{
  \textbf{Participant i}  \> \textbf{Other Participants}  \\
  \uncover<1-> { \secretone_i  \rgets \Zq \\ }
  \uncover<2-> { (\{ \shareone_{i1}, \ldots, \shareone_{in}\}, \{\secretone_i, a_{i1}, \ldots, a_{i(t-1)} \}) \rgets \shamir.\issueshares(\secretone_i, n, t) \> \\ }
  \uncover<3-> { \vsscommitone_i  = \langle A_{i0}, \ldots, A_{i(t-1)} \rangle \gets \shamir.\commit(\secretone_i, a_{i1}, \ldots, a_{i(t-1)}) \> \\ }
  \uncover<4-> { \> \sendmessageright{top=\text{ {\bf Broadcast}  $\vsscommitone_i $   }}  \< \\}
  \uncover<5-> { \> \sendmessageright{top=\text{ {\bf Send} $\shareone_{ij} $   }}  \< \\ }
  \uncover<6-> {  \shamir.\verify(\shareone_i, \vsscommitone_j) \iseq 1  \> \\ }
  \uncover<7-> { \sk_i = \sum_{k \in \qual} \shareone_k,\ \ \text{$\qual$ are the players that issued valid shares. }    \\ }
  \uncover<8-> {  \pk \gets  \prod_{k \in \qual} A_{k0}  = g^{\sum_{i \in \qual} \secretone_i  }   \> \\ }
}
}
};
\end{tikzpicture}
\end{frame}

\begin{frame}
  \frametitle{Key Bias Attack: High Level}

  \begin{itemize}
    \item Performed by a \emph{rushing adversary}, the adversary that ``speaks last'' in the protocol.
    \item[]
    \item<2-> In Pedersen DKG, a rushing adversary can learn the output public key before other participants.
    \item[]
    \item<3-> Nothing prevents a rushing adversary from adaptively choosing their contributions to bias key material.
  \end{itemize}
\end{frame}

\begin{frame}
  \centering
\begin{tikzpicture}
\node at (0,0) {
\scalebox{0.9} {
  \procedure{Key Bias Attack by a Rushing Adversary}{
  \textbf{Adversary}  \> \textbf{Other Participants}  \\
  \uncover<1-> { \> \sendmessageleft{top=\text{ {\bf Broadcast}  $\vsscommitone_i$   }}   \\}
  \uncover<2-> { \> \sendmessageleft{top=\text{ {\bf Send} $\shareone_{ij}$   }}   \\ }
  \uncover<3-> { \text{Until $\pk$ is even, do:}   } \\
  \uncover<4-> { \ \ \ \ \ \ \secretone_A  \rgets \Zq;\ \text{ check }  \pk = g^{\secretone_A} \cdot  \prod_{k =1; k \neq A}^n A_{k0}    \\ }
  \uncover<5-> { (\{ \shareone_{A1}, \ldots, \shareone_{An}\}, \{\secretone_A, a_{A1}, \ldots, a_{A(t-1)} \}) \rgets \shamir.\issueshares(\secretone_A, n, t) \> \\ }
  \uncover<6-> { \vsscommitone_A  = \langle A_{A0}, \ldots, A_{A(t-1)} \rangle \gets \shamir.\commit(\secretone_A, a_{A1}, \ldots, a_{A(t-1)}) \> \\ }
  \uncover<7-> { \> \sendmessageright{top=\text{ {\bf Broadcast}  $\vsscommitone_A $   }}  \< \\}
  \uncover<8-> { \> \sendmessageright{top=\text{ {\bf Send} $\shareone_{Aj} $   }}  \< \\ }
}
}
};
\end{tikzpicture}
\end{frame}


\begin{frame}
  \huge
  \centering
  GJKR Construction
\end{frame}

\begin{frame}
  \frametitle{Overview}

  \begin{itemize}
    \item Prevents the key-bias attack by using blinded commitments and verifiable extraction techniques.
    \item[]
    \item<2-> Participants issue a blinded VSS commitment in round one and then unblind this commitment in round three.
    \item[]
    \item<3-> Share correctness is verified with respect to the blinded commitment.
    \item[]
    \item<4-> If the adversary cheats in the last round, their contribution can be extracted by honest players (ensuring key material remains unbiased).
  \end{itemize}
\end{frame}

\begin{frame}
  \centering
\begin{tikzpicture}
\node at (0,0) {
\scalebox{0.9} {
  \procedure{GJKR Construction: Round One}{
    \textbf{\underline{Participant i}}  \> \textbf{\underline{Other Participants}}  \\
  \uncover<1-> { \secretone_i, b_i  \rgets \Zq \\ }
  \uncover<2-> { (\{ \shareone_{i1}, \ldots, \shareone_{in}\}, \{\secretone_i, a_{i1}, \ldots, a_{i(t-1)} \}) \rgets \shamir.\issueshares(\secretone_i, n, t) \> \\ }
  \uncover<3-> { \vsscommitone_i  = \langle A_{i0}, \ldots, A_{i(t-1)} \rangle \gets \shamir.\commit(\secretone_i, a_{i1}, \ldots, a_{i(t-1)}) \> \\ }
  \uncover<4-> { (\{ \sharetwo_{i1}, \ldots, \sharetwo_{in}\}, \{b_i, \hat{a_{i1}}, \ldots, \hat{a_{i(t-1)}} \}) \rgets \shamir.\issueshares(b_i, n, t) \> \\ }
  \uncover<5-> { \vsscommittwo_i  = \langle \hat{A_{i0}}, \ldots, \hat{A_{i(t-1)}} \rangle \gets \shamir.\commit(b_i, \hat{a_{i1}}, \ldots, \hat{a_{i(t-1)}}) \> \\ }
  \uncover<6-> { \vsscommitthree_i  = \langle B_{i0}, \ldots, B_{i(t-1)} \rangle  \gets \langle A_{i0}\hat{A_i0}, \ldots,  A_{i(t-1)}\hat{A_i(t-1)}     \rangle \\ }
%
  \uncover<7-> { \qquad \qquad \sendmessageright{top=\text{ {\bf Broadcast}  $\vsscommitthree_i $   }}  \< \\}
  \uncover<8-> { \qquad \qquad  \sendmessageright{top=\text{ {\bf Send} $\shareone_{ij},  \sharetwo_{ij}$ to player $j$   }}  \< \\ }
  \uncover<9-> { \text{Check } g^{\shareone_{ij}}h^{\sharetwo_{ij}} \text{ is correct with respect to } \vsscommitthree_i \> \\ }
  \uncover<10-> { \qual \text{ are the set of players where this check holds}   \> \\ }
}
}
};
\end{tikzpicture}
\end{frame}


\begin{frame}
  \centering
\begin{tikzpicture}
\node at (0,0) {
\scalebox{0.9} {
  \procedure{GJKR Construction: Rounds Two and Three}{
    \textbf{\underline{Participant i}}  \> \textbf{\underline{Other Participants}}  \\
  \uncover<1-> { \qquad \qquad \sendmessageright{top=\text{ {\bf Broadcast}  $\vsscommitone_i $   }}  \< \\}
  \uncover<2-> { \text{Check } \shamir.\verify(\shareone_{ji}, \vsscommitone_j) \iseq 1  \> \\ }
  \uncover<3-> { \text{If any player $k$ failed, $t$ honest players combine shares to recover $\secretone_k$  }  \> \\ }
%
  \uncover<4-> { \sk_i = \sum_{k \in \qual} \shareone_k    \\ }
  \uncover<5-> {  \pk \gets  \prod_{k \in \qual} A_{k0}  = g^{\sum_{i \in \qual} \secretone_i  }   \> \\ }
}
}
};
\end{tikzpicture}
\end{frame}


\begin{frame}
  \frametitle{Takeaways}

  \begin{itemize}
    \item Distributed Key Generation protocols are a useful building block for distributing trust among a set of authorized parties.
    \item[]
    \item<2-> Properties taken for granted in the centralized model, such as uniform distribution of key material, are difficult to guarantee in a multi-party setting.
    \item[]
    \item<3-> Pedersen DKG is a simple DKG whose security is proven in some contexts but not in others.
    \item[]
    \item<4-> The GJKR construction is proven in a stand-alone manner, but requires more network rounds and computational complexity.
  \end{itemize}
\end{frame}

\begin{frame}

  \huge
  \centering
  Thank you!
\end{frame}

\end{document}


